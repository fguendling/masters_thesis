\chapter{Grundlagen}
\label{grundlagen}


%\epigraph{Ein Verständnis der Möglichkeiten ist der Schlüssel […] zur Gestaltung und Durchführung von Text-Analysen.}{--- \textup{aus} \cite[S. vii]{Anandarajan}}

\section{Künstliche Intelligenz, Machine Learning \& Deep Learning}


Als Ontologien festgelegt sind im Umfeld der vorliegenden Evaluation beispielsweise die Ontologien ``Architectural Language``, ``Requirements``, ``Design``, ``Logistics``, ``Generic``, die sich jeweils aus verschiedenen Begriffen zusammensetzen, die das Thema der jeweiligen Ontologie beschreiben. Die Begriffe und Begriffspaare können auch gewichtet werden, um zum Ausdruck zu bringen, wie stark der Bezug eines Begriffs wie etwa Capability zu einem Topic wie ARCH ist. Verwendet werden können die Begriffe und die Gewichtungen etwa für eine regelbasierte Themenerkennung. Wenn bestimmte Begriffe einer Ontologie häufig in auszuwertenden Textabschnitten vorkommen, dann erfolgt die Einteilung des entsprechenden Textabschnitts zu einer der festgelegten Ontologien. 
